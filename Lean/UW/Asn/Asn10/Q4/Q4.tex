% filepath: c:\Users\sayed\School\1-B\ECE-108\Lean\UW\Asn\Asn10\Q4\Q4.tex
\documentclass{article}
\usepackage{amsmath, amssymb}

\title{ECE-108 Assignment 10: Analysis of Covid-19 Rapid Test}
\author{Sayed Hayat Ahmad, Robert Yin}
\date{\today}

\begin{document}

\maketitle

\section{Analysis of Covid-19 Rapid Test}
\textbf{Given information:}
\begin{itemize}
    \item The Covid-19 rapid test has:
    \begin{itemize}
        \item Sensitivity = 78\%
        \item Specificity = 99\%
    \end{itemize}
    \item First scenario: Covid-19 actual prevalence = 10\%
    \item Second scenario: Covid-19 actual prevalence = 1\%
\end{itemize}

\subsection*{Q4a: Positive Results (10\% prevalence)}

What percentage of people test positive for Covid-19?

\subsubsection*{Solution}

We need to calculate the percentage of people who test positive.

Total positive test results = True positives + False positives
\begin{align}
\text{True positives} &= \text{Sensitivity} \times \text{Prevalence} = 0.78 \times 0.10 = 0.078 \text{ (7.8\%)}\\
\text{False positives} &= (1 - \text{Specificity}) \times (1 - \text{Prevalence}) = 0.01 \times 0.90 = 0.009 \text{ (0.9\%)}\\
\text{Total positive tests} &= 0.078 + 0.009 = 0.087 \text{ (8.7\%)}
\end{align}

\fbox{Answer: 8.7\% of people test positive for Covid-19.}

\subsection*{Q4b: False Negatives (10\% prevalence)}

What percentage of people get a false negative result from the test?

\subsubsection*{Solution}

False negatives occur when infected people test negative:
\begin{align}
\text{False negative rate} &= (1 - \text{Sensitivity}) \times \text{Prevalence}\\
&= 0.22 \times 0.10 = 0.022 \text{ (2.2\%)}
\end{align}

\fbox{Answer: 2.2\% of people get a false negative result from the test.}

\subsection*{Q4c: Estimated Prevalence (10\% prevalence)}

Based on the test results, what is the estimated prevalence?

\subsubsection*{Solution}

The estimated prevalence is the percentage of people who test positive:

\fbox{Answer: The estimated prevalence is 8.7\%.}

\subsection*{Q4d: Analysis of Estimated Prevalence}

Why does the estimated prevalence differ from the actual prevalence?

\subsubsection*{Solution}

The estimated prevalence (8.7\%) is lower than the actual prevalence (10\%) because:
\begin{itemize}
    \item The test's sensitivity isn't perfect (78\%)
    \item This means 22\% of actual positive cases are missed (false negatives)
    \item Even though there are some false positives, they don't fully compensate for the missed cases
\end{itemize}

\section*{Repeat for an actual prevalence of 1\%}

\subsection*{Q4e: Positive Results (1\% prevalence)}

What percentage of people test positive for Covid-19?

\subsubsection*{Solution}

Total positive test results = True positives + False positives
\begin{align}
\text{True positives} &= \text{Sensitivity} \times \text{Prevalence} = 0.78 \times 0.01 = 0.0078 \text{ (0.78\%)}\\
\text{False positives} &= (1 - \text{Specificity}) \times (1 - \text{Prevalence}) = 0.01 \times 0.99 = 0.0099 \text{ (0.99\%)}\\
\text{Total positive tests} &= 0.0078 + 0.0099 = 0.0177 \text{ (1.77\%)}
\end{align}

\fbox{Answer: 1.77\% of people test positive for Covid-19.}

\subsection*{Q4f: False Negatives (1\% prevalence)}

What percentage of people get a false negative result from the test?

\subsubsection*{Solution}

False negatives occur when infected people test negative:
\begin{align}
\text{False negative rate} &= (1 - \text{Sensitivity}) \times \text{Prevalence}\\
&= 0.22 \times 0.01 = 0.0022 \text{ (0.22\%)}
\end{align}

\fbox{Answer: 0.22\% of people get a false negative result from the test.}

\subsection*{Q4g: Estimated Prevalence (1\% prevalence)}

Based on the test results, what is the estimated prevalence?

\subsubsection*{Solution}

The estimated prevalence is the percentage of people who test positive:

\fbox{Answer: The estimated prevalence is 1.77\%.}

\subsection*{Q4h: Analysis of Effect of Prevalence}

Why does the actual prevalence affect the difference between the actual and estimated prevalence?

\subsubsection*{Solution}

The actual prevalence affects the difference between actual and estimated prevalence because:
\begin{itemize}
    \item At lower prevalence (1\%), false positives become more significant relative to true positives
    \item This causes the estimated prevalence (1.77\%) to be higher than the actual (1\%)
    \item At higher prevalence (10\%), false negatives become more significant 
    \item This causes the estimated prevalence (8.7\%) to be lower than the actual (10\%)
    \item The balance point depends on the test's sensitivity and specificity values
\end{itemize}

\end{document}