% filepath: c:\Users\sayed\School\1-B\ECE-108\Lean\UW\Asn\Asn10\Q3\Q3.tex
\documentclass{article}
\usepackage{amsmath, amssymb}

\title{ECE-108 Assignment 10: Resistor Problem}
\author{Sayed Hayat Ahmad, Robert Yin}
\date{\today}

\begin{document}

\maketitle

\section{Resistors}

\subsection*{Problem}

An electrical engineer tests a batch of 20 resistors and finds that 15\% of them are outside the specified tolerance. If 3 resistors are randomly selected from this batch, what is the probability that exactly 1 of them is outside the specified tolerance?

\subsection*{Solution}

Given information:
\begin{itemize}
    \item Batch contains 20 resistors
    \item 15\% of the resistors are outside the specified tolerance
    \item We randomly select 3 resistors from the batch
\end{itemize}

First, let's determine the number of resistors outside tolerance:
\begin{align}
\text{Number outside tolerance} &= 20 \times 0.15 = 3\\
\text{Number within tolerance} &= 20 - 3 = 17
\end{align}

This is a hypergeometric distribution problem since we're sampling without replacement from a finite population. We need to find the probability of selecting exactly 1 resistor that is outside tolerance when randomly choosing 3 resistors.

Using the hypergeometric probability formula:
\begin{align}
P(X = 1) &= \frac{\binom{3}{1} \times \binom{17}{2}}{\binom{20}{3}}
\end{align}

Where:
\begin{itemize}
    \item $\binom{3}{1}$ = ways to select 1 defective from the 3 defective resistors
    \item $\binom{17}{2}$ = ways to select 2 good from the 17 good resistors
    \item $\binom{20}{3}$ = total ways to select 3 resistors from 20
\end{itemize}

Calculating each term:
\begin{align}
\binom{3}{1} &= 3\\
\binom{17}{2} &= \frac{17!}{2! \times 15!} = \frac{17 \times 16}{2} = 136\\
\binom{20}{3} &= \frac{20!}{3! \times 17!} = \frac{20 \times 19 \times 18}{6} = 1140
\end{align}

Therefore:
\begin{align}
P(X = 1) &= \frac{3 \times 136}{1140}\\
&= \frac{408}{1140}\\
&= 0.3579
\end{align}

\fbox{Answer: The probability is 35.8\%.}

\end{document}