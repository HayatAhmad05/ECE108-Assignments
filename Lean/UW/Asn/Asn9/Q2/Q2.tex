\documentclass{article}
\usepackage{amsmath, amssymb, amsthm}
\usepackage{graphicx}
\usepackage{fancyhdr}
\usepackage{hyperref}
\usepackage{enumitem}
\usepackage[margin=1in]{geometry}

\title{ECE 108 - Assignment 9}
\author{Sayed Hayat Ahmad, Robert Yin}
\date{\today}

\begin{document}
\maketitle

\section*{Question 2a}

You are verifying the correctness of floating-point divider circuits.  Your first task is to figure out how large of a circuit (as measured in the number of bits in the input data) you can verify.

\begin{itemize}
    \item The circuit has 2 inputs, each of $n$ bits, where $n$ is the value that you need to determine.
    
    \item It takes 2000 clock cycles on a computer to simulate one test case.
    
    \item Your computers have a clock speed of 3.5 GHz.
    
    \item You have a rack of 200 computers to use for simulation.
    
    \item You may run the simulations for 3 months (90 days).
\end{itemize}

What is the maximum value of $n$ that satisfies your constraints?

\section*{Solution}
The maximum value of $n$ that satisfies the given constraints is 25 bits.

To arrive at this conclusion, let's break down the problem:

\textbf{Total available clock cycles:}
\begin{align*}
\text{Clock speed} &= 3.5 \text{ GHz} = 3.5 \times 10^9 \text{ Hz} \\
\text{Number of computers} &= 200 \\
\text{Simulation duration} &= 90 \text{ days} = 90 \times 24 \times 60 \times 60 \text{ seconds} \\
\text{Total clock cycles} &= 3.5 \times 10^9 \times 200 \times 90 \times 24 \times 60 \times 60 \\
&= 5.4432 \times 10^{18} \text{ cycles}
\end{align*}

\textbf{Maximum number of test cases:}
\begin{align*}
\text{Each test case requires} &= 2000 \text{ clock cycles} \\
\text{Maximum test cases} &= \frac{5.4432 \times 10^{18}}{2000} \approx 2.7216 \times 10^{15} \text{ test cases}
\end{align*}

\textbf{Determining the maximum value of $n$:}
\begin{align*}
\text{Each input is $n$ bits, and there are 2 inputs} \\
\text{Total possible combinations} &= 2^{2n} \\
\text{We need } 2^{2n} &\leq 2.7216 \times 10^{15} \\
\text{Solving this inequality: } n &\leq \log_2(\sqrt{2.7216 \times 10^{15}}) \\
\end{align*}

The floor of this value gives us the maximum integer $n$: 25

\section*{Question 2b}

If the circuit has 64 bit inputs, how long will it take to run the verification?

\section*{Solution}

To determine how long it will take to run verification with 64-bit inputs, we need to calculate:

\textbf{Total number of test cases:}
\begin{align*}
\text{With two 64-bit inputs} &= 2^{(2 \times 64)} = 2^{128} \text{ possible input combinations} \\
&= 3.40282 \times 10^{38} \text{ test cases}
\end{align*}

\textbf{Total available computational power:}
\begin{align*}
\text{Clock speed} &= 3.5 \text{ GHz} = 3.5 \times 10^9 \text{ cycles per second} \\
\text{Number of computers} &= 200 \\
\text{Simulation duration} &= 90 \text{ days} = 90 \times 24 \times 60 \times 60 \text{ seconds} \\
\text{Total available clock cycles} &= 3.5 \times 10^9 \times 200 \times 90 \times 24 \times 60 \times 60 \\
&= 5.4432 \times 10^{18} \text{ cycles}
\end{align*}

\textbf{Time required for verification:}
\begin{align*}
\text{Each test case requires} &= 2000 \text{ clock cycles} \\
\text{Total clock cycles needed} &= 3.40282 \times 10^{38} \times 2000 \\
&= 6.80564 \times 10^{41} \text{ cycles} \\
\text{Time required (in seconds)} &= \frac{6.80564 \times 10^{41}}{5.4432 \times 10^{18}} \\
&= 1.25031 \times 10^{23} \text{ seconds} \\
\text{Time required (in years)} &= \frac{1.25031 \times 10^{23}}{60 \times 60 \times 24 \times 365} \\
&= 3.96468 \times 10^{15} \text{ years}
\end{align*}

Therefore, it would take approximately $3.96 \times 10^{15}$ years to complete the verification of a circuit with two 64-bit inputs.

\end{document}