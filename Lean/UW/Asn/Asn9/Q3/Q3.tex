\documentclass{article}
\usepackage{amsmath, amssymb, amsthm}
\usepackage{graphicx}
\usepackage{fancyhdr}
\usepackage{hyperref}
\usepackage{enumitem}
\usepackage[margin=1in]{geometry}

\title{ECE 108 - Assignment 9}
\author{Sayed Hayat Ahmad, Robert Yin}
\date{\today}

\begin{document}
\maketitle

\section*{Question 3a}

If you want to test all possible configurations of devices connected to the bus, how many configurations satisfy the criteria below?

\begin{itemize}
    \item All slots are identical -- that is, a configuration is dependent upon only which devices are connected to the bus, independent of which slot a device is in.
    \item Empty slots are allowed.
\end{itemize}

\begin{itemize}
    \item \textbf{Multiple} devices of the \textbf{same category} from \textbf{different manufacturers} may be used.
    \item \textbf{At most one} instance of the same device (\textbf{same category} and \textbf{same manufacturer}) may be used.
    \item Multiple \textbf{different categories} from the \textbf{same manufacturer} may be used.
\end{itemize}

\section*{Solution}

To find the number of configurations that satisfy the criteria:

Each unique device (specific category and manufacturer combination) has two possibilities:
\begin{itemize}
    \item Include it once
    \item Don't include it
\end{itemize}

Since we can choose independently for each unique device, the total number of configurations is:
\begin{align}
    2^n
\end{align}

Where $n =$ total number of unique devices 

For example, with 3 categories and 4 manufacturers, there would be 12 unique devices, giving $2^{12} = 4,096$ possible configurations.


\section*{Question 3b}

If you want to test all possible configurations of devices connected to the bus, how many configurations satisfy the criteria below?

\begin{itemize}
    \item All slots are identical 
    \item Each slot has a device (no empty slots)
\end{itemize}

\begin{itemize}
    \item \textbf{At most one} device from \textbf{each category} may be used.
    \item Multiple \textbf{different categories} from the \textbf{same manufacturer} may be used.
\end{itemize}

\section*{Solution}

To find the number of configurations that satisfy the criteria:

For each category, we have the following options:
\begin{itemize}
    \item Choose one device from any manufacturer
    \item Don't use any device from this category
\end{itemize}

The number of choices for each category is $(m + 1)$, where $m$ is the number of manufacturers.

Using the multiplication principle of counting, the total number of possible configurations is:
\begin{align}
    (m + 1)^c
\end{align}

Where:
\begin{itemize}
    \item $m =$ number of manufacturers
    \item $c =$ number of categories
\end{itemize}

However, since empty slots are not allowed (each slot must have a device), we must exclude the configuration where no devices are selected.

Therefore, the final answer is:
\begin{align}
    (m + 1)^c - 1
\end{align}

For example, with 3 manufacturers and 4 categories, we would have $(3 + 1)^4 - 1 = 4^4 - 1 = 256 - 1 = 255$ possible configurations.



\section*{Question 3c}

If we modify Q3b to allow empty slots, the problem does not fit any of our standard techniques. Explain why.


\section*{Solution}

The modified problem doesn't fit standard combinatorial techniques because it creates a complex relationship between two factors:

\begin{enumerate}
    \item The selection of devices (at most one per category)
    \item The arrangement of these devices in slots (with some slots potentially empty)
\end{enumerate}

This creates a dependency that's difficult to model with basic formulas:
\begin{itemize}
    \item The number of ways to arrange devices depends on how many we select
    \item The number of valid device selections depends on how many slots we have
\end{itemize}

Unlike the previous version where we could simply count device selections, we now need to consider both selection and arrangement simultaneously. This makes the problem more complex and requires a different approach to solve.







\end{document}